%% LyX 2.0.6 created this file.  For more info, see http://www.lyx.org/.
%% Do not edit unless you really know what you are doing.
\documentclass[12pt,british,12pt, round,comma,sort&compress]{article}
\usepackage{mathptmx}
\renewcommand{\familydefault}{\rmdefault}
%\usepackage[T1]{fontenc}
%\usepackage[latin9]{inputenc}
\usepackage[a4paper]{geometry}
\usepackage{graphicx}
\geometry{verbose,tmargin=2cm,bmargin=2cm,lmargin=2cm,rmargin=2cm}
\setcounter{secnumdepth}{5}
\setcounter{tocdepth}{5}
\setlength{\parskip}{\smallskipamount}
\setlength{\parindent}{0pt}

\makeatletter
%%%%%%%%%%%%%%%%%%%%%%%%%%%%%% User specified LaTeX commands.
\usepackage{color}
\newcommand{\nicefrac}[2]{\ensuremath ^{#1}\!\!/\!_{#2}}
\usepackage { fancybox}

\renewcommand{\floatpagefraction}{0.95}
\renewcommand{\textfraction}{0}
\renewcommand{\topfraction}{1}
\renewcommand{\bottomfraction}{1}

\usepackage{bibentry}
\bibliographystyle{abbrvnat}
\nobibliography{numerics}

\makeatother

\usepackage{babel}
\begin{document}

\title{Literature review relevant for Advection}
\author{Hilary Weller}
\maketitle

\begin{itemize}

\item \bibentry{NdFG22} \ \\

\item \bibentry{Gass21} \ \\

\item \bibentry{SM98} \ \\
Sign preserving rather than monotone.\\
``MPDATA effectively limits the magnitude of the vector velocity''\\
``most common complaints are that the basic MPDATA is too diffusive, and enhanced MPDATA is too expensive.''

\item \bibentry{KS17} \ \\
``pseudo-velocity of the scheme to rely only on face-normal advective fluxes
to the dual cells, in contrast to the full vector employed in previous implementations''
\\
``earlier developments, such as the nonoscillatory option, the infinite-
gauge variant, and moving curvilinear meshes''

\item \bibentry{SG90} \ \\
``merges the flux-corrected transport (FCT) methodology of Boris and Book [4-S] and Zalesak [7] with the iterative formalism of MPDATA.''

\item \bibentry{SC86} \ \\
Adding a constant to the tracer makes it more accurate

\item\bibentry{Smol83}

\item \bibentry{MS06}

\item \bibentry{KNK15}\ \\
4th order, 2D scheme versus 5th order 1d WENO scheme. No discussion of how the splitting is done - but it formally limits the scheme to 2nd order. Although results for both schemes are ~3-4th order in test cases

\item \bibentry{SM89} \ \\
non-negative results for $c>1$ and high accuracy for $c<1$. \\
Zalesak's flux limiter \\
HO fluxes are 4th order time, space-centred.\\
FD. Cheaper than explicit for $c>1$. 1D results

\item \bibentry{Kuz09} \ \\
``In order to reduce the cost of flux correction, the raw antidiffusive fluxes
are linearized about an auxiliary solution computed by a high- or low-order scheme. By virtue of this linearization, the costly computation of solution-dependent correction factors is to be performed just once per time step'' \\
``Three FEM-FCT schemes based on the Runge–Kutta, Crank–Nicolson, and backward Euler time-stepping''. \\
Tests with Courant number up to 6.4

For $\theta$ method there are restrictions on the time step related to the Courant number and $\theta$.\\
Explicict TVD RK methods have no negative coefficients.

\item \bibentry{VBV08} \ \\

\item \bibentry{ZWS06}
\item \bibentry{XL01}
\item \bibentry{Smo82}
\item \bibentry{LR05b}\ \\
\includegraphics[width=0.49\linewidth]{handNotes/LR05b_1.jpg}
\includegraphics[width=0.49\linewidth]{handNotes/LR05b_2.jpg}


\item \bibentry{LLM96}\\
\includegraphics[scale=0.5]{handNotes/LLM96.jpg}

\item \bibentry{SP92}\\
Convergence criterion for SL:
$||\partial\mathbf{v}/\partial\mathbf{x}||\Delta t < 1$

\item \bibentry{PS84}\\
A sufficient condition for convergence is
$\Delta t \max(|u_x|, |u_y|, |v_x|, |v_y|) < 1$

\item \bibentry{LLM95}

\item \bibentry{CW84}\\
PPM in 1d, variable resolution. Reconstruct quartic polynomial over each cell using cell area values. Then limit the end points to avoid over-shoots and undershoots. Semi-Lagrangian in nature, predicting cell average values rather than fluxes. If solution is smooth, 4th order accurate. 3rd order otherwise\\
\includegraphics[width=0.49\linewidth]{handNotes/CW84_1.jpg}
\includegraphics[width=0.49\linewidth]{handNotes/CW84_2.jpg}

\item \bibentry{BS99}\\
\includegraphics[scale=0.5]{handNotes/BS99.jpg}

\item \bibentry{PL07}\\
Direction splitting to extend PPM no a non-orthogonal cubed-sphere grid. Uses co-variant and contra-variant velocity components. Final scheme comes out as 2nd order\\
\includegraphics[scale=0.5]{handNotes/PL07.jpg}

\item \bibentry{Thu96b}
\item \bibentry{Miu07b}\ \\
\includegraphics[scale=0.33]{handNotes/Miu07b}
\item \bibentry{Miu13}\ \\
\includegraphics[scale=0.33]{handNotes/Miu13}
\item \bibentry{LR96}\\
Direction splitting to use PPM on a lat-lon grid\\

\item \bibentry{Nor14}
\item \bibentry{IMBO14}
\item \bibentry{HAQ12}
\item \bibentry{CLSX14}
\item \bibentry{CCN+07}
%\item \bibentry{BS14}
\item \bibentry{Lau07} \ \\
\includegraphics[scale=0.5]{handNotes/Lau07.jpg}

\item \bibentry{Zal79}
\item \bibentry{SM10}\ \\
\includegraphics[scale=0.5]{handNotes/SM10.jpg}\\
\includegraphics[scale=0.5]{handNotes/SM10_2.jpg}

\item \bibentry{SG11}\ \\
\includegraphics[scale=0.5]{handNotes/SG11.jpg}

\item \bibentry{Bott10}\ \\
\includegraphics[scale=0.33,angle=270]{handNotes/Bott10.jpg}

\item \bibentry{LML93}\ \\
Multi-dimensional flux limiters\\
Flow in a constant direction oblique to a grid is more difficult that around in a circle.

\item \bibentry{vanSlin07} \ \\
Defines a local theta scheme

\item \bibentry{GST01} \ \\
``construction of optimal explicit SSP linear Runge-Kutta methods'' \\
``SSP property of implicit Runge-Kutta and multistep methods'' \\
``It is a common practice in solving time-dependent partial differential equations (PDEs) to first discretize the spatial variables to obtain a semidiscrete method of lines scheme. This is then an ordinary differential equation (ODE) system in the time variable, which can be discretized by an ODE solve.''\\
SSP equivalent to TVD.\\
``m-stage, mth-order SSP Runge-Kutta method can have, at most, CFL coefficient
c = 1'' \\
``version of Harten's lemma [8] for the TVD property of implicit backward Euler
method'' (First-order upwind in space, backward Euler is TVD).
``nonexistence of SSP implicit Runge-Kutta or multistep methods of order higher than 1''.

\item \bibentry{YWH85} \ \\
``guaranteed not to generate spurious oscillations for a nonlinear scalar equation and a constant coefficient system''
This papers applies Harten's implicit TVD schemes to steady state calculations.\\
``backward Euler TVD schemes for the initial value problem. ... This is the only unconditionally stable TVD scheme belonging to the one-parameter family of TVD schemes considered in [3].'' \\
``review the proof that the backward Euler scheme is unconditionally TVD.'' \\
``Truncation error analysis shows that the first-order accurate scheme (2.12) is a
second-order accurate approximation to solutions of the modified equation''\\
``Linearized Version of the implicit TVD Scheme''\\
``The nonlinear, spatially second-order accurate, unconditionally stable implicit
TVD scheme in a linearized nonconservative form has been applied to obtain
steady-state solutions for the one-dimensional compressible inviscid equations of
gas dynamics.''\\
``only conservative after the solution reaches steady state''

\item \bibentry{Hart83} \ \\
Actually this is all about explicit schemes.

\item \bibentry{YWH82} \ \\
This is the technical report about implicit TVD schemes.

\item \bibentry{Yee87} \ \\
``when steady-state calculations are sought, the numerical solution is independent
of the time step.''\\
For constant coefficients, slope limiters and flux limiters are equivalent.\\
``symmetric (or non-upwind) TVD scheme''\\
``As of this writing, the conservative linearlized form (3.30) has not been proven to be TVD. Yet numerical study shows that for moderate CFL numbers, (3.30) produces high-resolution shocks and nonoscillatory solutions.''\\

\item \bibentry{LZ22} \ \\
``the adaptively implicit advection algorithm allows a larger time step in the presence of certain extreme conditions featured by strong vertical velocity. This benefits the use of higher vertical resolutions and relatively larger time-step choice''\\
``slightly more diffusive than the fully explicit advection''\\
``reduce the strength of certain maximum rainfall''.\\
``Because we focus on moist tracer transport, for which monotonicity is highly desirable, only the first-order upwind scheme is used for the discretization of
the numerical flux of the implicit component.''\\
GRIST model -- 2 dimensionally unstructured C-grid.\\
Test cases:\\
Hadley‑like meridional circulation\\
Three‑dimensional deformational flow\\
Splitting supercell test case\\
Tracer-tracer correlations\\
Real-world global simulations\\


\end{itemize}

\nobibliography{numerics}

\end{document}
