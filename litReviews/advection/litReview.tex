%% LyX 2.0.6 created this file.  For more info, see http://www.lyx.org/.
%% Do not edit unless you really know what you are doing.
\documentclass[12pt,british,12pt, round,comma,sort&compress]{article}
\usepackage{mathptmx}
\renewcommand{\familydefault}{\rmdefault}
%\usepackage[T1]{fontenc}
%\usepackage[latin9]{inputenc}
\usepackage[a4paper]{geometry}
\usepackage{graphicx}
\geometry{verbose,tmargin=2cm,bmargin=2cm,lmargin=2cm,rmargin=2cm}
\setcounter{secnumdepth}{5}
\setcounter{tocdepth}{5}
\setlength{\parskip}{\smallskipamount}
\setlength{\parindent}{0pt}

\makeatletter
%%%%%%%%%%%%%%%%%%%%%%%%%%%%%% User specified LaTeX commands.
\usepackage{color}
\newcommand{\nicefrac}[2]{\ensuremath ^{#1}\!\!/\!_{#2}}
\usepackage { fancybox}

\renewcommand{\floatpagefraction}{0.95}
\renewcommand{\textfraction}{0}
\renewcommand{\topfraction}{1}
\renewcommand{\bottomfraction}{1}

\usepackage{bibentry}
\bibliographystyle{abbrvnat}
\nobibliography{numerics}

\makeatother

\usepackage{babel}
\begin{document}

\title{Literature review relevant for Advection}
\author{Hilary Weller}
\maketitle

\begin{itemize}

\item \bibentry{KNK15}\ \\
4th order, 2D scheme versus 5th order 1d WENO scheme. No discussion of how the splitting is done - but it formally limits the scheme to 2nd order. Although results for both schemes are ~3-4th order in test cases

\item \bibentry{ZWS06}
\item \bibentry{XL01}
\item \bibentry{Smo82}
\item \bibentry{LR05b}\ \\
\includegraphics[width=0.49\linewidth]{handNotes/LR05b_1.jpg}
\includegraphics[width=0.49\linewidth]{handNotes/LR05b_2.jpg}


\item \bibentry{LLM96}\\
\includegraphics[scale=0.5]{handNotes/LLM96.jpg}

\item \bibentry{SP92}\\
Convergence criterion for SL:
$||\partial\mathbf{v}/\partial\mathbf{x}||\Delta t < 1$

\item \bibentry{PS84}\\
A sufficient condition for convergence is
$\Delta t \max(|u_x|, |u_y|, |v_x|, |v_y|) < 1$

\item \bibentry{LLM95}

\item \bibentry{CW84}\\
PPM in 1d, variable resolution. Reconstruct quartic polynomial over each cell using cell area values. Then limit the end points to avoid over-shoots and undershoots. Semi-Lagrangian in nature, predicting cell average values rather than fluxes. If solution is smooth, 4th order accurate. 3rd order otherwise\\
\includegraphics[width=0.49\linewidth]{handNotes/CW84_1.jpg}
\includegraphics[width=0.49\linewidth]{handNotes/CW84_2.jpg}

\item \bibentry{BS99}\\
\includegraphics[scale=0.5]{handNotes/BS99.jpg}

\item \bibentry{PL07}\\
Direction splitting to extend PPM no a non-orthogonal cubed-sphere grid. Uses co-variant and contra-variant velocity components. Final scheme comes out as 2nd order\\
\includegraphics[scale=0.5]{handNotes/PL07.jpg}

\item \bibentry{Thu96b}
\item \bibentry{Miu07b}\ \\
\includegraphics[scale=0.33]{handNotes/Miu07b}
\item \bibentry{Miu13}\ \\
\includegraphics[scale=0.33]{handNotes/Miu13}
\item \bibentry{LR96}\\
Direction splitting to use PPM on a lat-lon grid\\

\item \bibentry{Nor14}
\item \bibentry{IMBO14}
\item \bibentry{HAQ12}
\item \bibentry{CLSX14}
\item \bibentry{CCN+07}
\item \bibentry{BS14}
\item \bibentry{Lau07} \ \\
\includegraphics[scale=0.5]{handNotes/Lau07.jpg}

\item \bibentry{Zal79}
\item \bibentry{SM10}\ \\
\includegraphics[scale=0.5]{handNotes/SM10.jpg}\\
\includegraphics[scale=0.5]{handNotes/SM10_2.jpg}

\item \bibentry{SG11}\ \\
\includegraphics[scale=0.5]{handNotes/SG11.jpg}

\item \bibentry{Bott10}\ \\
\includegraphics[scale=0.33,angle=270]{handNotes/Bott10.jpg}

\item \bibentry{LML93}\ \\
Multi-dimensional flux limiters\\
Flow in a constant direction oblique to a grid is more difficult that around in a circle.

\end{itemize}

\nobibliography{numerics}

\end{document}
