%% LyX 2.0.6 created this file.  For more info, see http://www.lyx.org/.
%% Do not edit unless you really know what you are doing.
\documentclass[12pt,british,12pt, round,comma,sort&compress]{article}
\usepackage{mathptmx}
\renewcommand{\familydefault}{\rmdefault}
%\usepackage[T1]{fontenc}
%\usepackage[latin9]{inputenc}
\usepackage[a4paper]{geometry}
\usepackage{graphicx}
\geometry{verbose,tmargin=2cm,bmargin=2cm,lmargin=2cm,rmargin=2cm}
\setcounter{secnumdepth}{5}
\setcounter{tocdepth}{5}
\setlength{\parskip}{\smallskipamount}
\setlength{\parindent}{0pt}

\makeatletter
%%%%%%%%%%%%%%%%%%%%%%%%%%%%%% User specified LaTeX commands.
\usepackage{color}
\newcommand{\nicefrac}[2]{\ensuremath ^{#1}\!\!/\!_{#2}}
\usepackage { fancybox}

\renewcommand{\floatpagefraction}{0.95}
\renewcommand{\textfraction}{0}
\renewcommand{\topfraction}{1}
\renewcommand{\bottomfraction}{1}

\usepackage{bibentry}
\bibliographystyle{abbrvnat}
\nobibliography{numerics}

\makeatother

\usepackage{babel}
\begin{document}

\title{Some Convection Parameterization Papers}
\author{Hilary Weller}
\maketitle

\subsection*{Convection parameterisation is a problem}
\begin{itemize}

\item 
Revcon proposal: \\
``in constructing a reliable and accurate parameterization underlie many current model deficiencies [IPCC, Ch. 8,10,12, K12, S10].''

\item \bibentry{IPCC5_WG1} \\
7.2.3.3 ``improved simulations of the Madden--Julian Oscillation, tropical convectively coupled waves and mean rain-fall patterns in some models ... but usually at the expense of a degraded simulation of the mean state.''

Other improvements since AR4. 

\item \bibentry{KS12} \\
Does not mention convection

\item \bibentry{SLF+10} \\
``The accumulations
from the four general circulation models that employ more
traditional parameterization of moist physics tend to produce
too much precipitation over the tropical oceans compared
to observations.''

The version of CAM with a traditional parameterization ``produces too much light rainfall that occurs as soon as the boundary layer energizes''.

\item \bibentry{HPB+14} \\
``Atmospheric convection is arguably the biggest current
obstacle to the improvement of global weather and
climate prediction (e.g. Sherwood et al., 2014).''

``Short-term forecasts of
global convection remain poor, with biases that have not
been removed by increasing resolution in current oper-
ational forecast models (Lin et al., 2012)''

\item \bibentry{SAB+13} \\
``Physical processes not well resolved by climate models continue to limit confidence in detailed predictions of climate change. The representation of cloud and convection-related processes dominates the model spread in global climate sensitivity, and affects the simulation of important aspects of the present-day climate especially in the tropics.''

\item \bibentry{ipcc41} \\
Chapter 8: Problems with MJO and TCs due to convection parametrization

\item \bibentry{LCD+08} \\
``It is shown that the 4- and 1-km-gridlength models often give more realistic-looking precipitation fields because convection is represented
explicitly rather than parameterized. However, the 4-km model representation suffers from large convective cells and delayed initiation because the grid length is too long to correctly reproduce the convection explicitly.''

\end{itemize}

\subsection*{Others}
\begin{itemize}
\item \bibentry{GR90}
\item \bibentry{AS74} \ \\ See notes in \url{notes/AS74.pdf}
\item \bibentry{KF90}
\item \bibentry{Kai04}
\item \bibentry{LR01} \\
Demonstrates the equivalence between using a higher order closure and predicting $w^\prime w^\prime$ and $w^\prime w^\prime w^\prime$ with predicting mass flux and area fraction, $\sigma$. Transfer terms not clear

\item \bibentry{LR01b} \ \\
Gives the entrainment and detrainment rates for the the previous paper. 

\item \bibentry{KH17} \ \\
\includegraphics[angle=90, width=\linewidth]{notes/KH17.jpg}

\end{itemize}

\subsection*{Recommended by Alan Grant on conditional averaging}
\begin{itemize}

\item \bibentry{SST07}\ \\
"(EDMF) is proposed for the dry
convective boundary layer. It is shown that the EDMF approach follows naturally from a decomposition of
the turbulent fluxes into 1) a part that includes strong organized updrafts, and 2) a remaining turbulent field.
At the basis of the EDMF approach is the concept that nonlocal subgrid transport due to the strong updrafts
is taken into account by the MF approach, while the remaining transport is taken into account by an ED closure."

"we define this strong updraft as a fixed fractional
area $a_u$ , say a few percent of the horizontal domain
under consideration, that contains the strongest upward
vertical velocities."

"convective mass flux can be defined as $M\equiv a_u (w_u - w)$. If we now
make use of the fact that $a_u << 1$.

\end{itemize}

\subsection*{Others on Convection}
\begin{itemize}

\item \bibentry{BFM16}\ \\
\begin{quote}
Observational studies show a strong correlation between large-scale wind convergence and precipitation. However, using this as a convective closure assumption to determine the total precipitation in a numerical model typically leads to deleterious wave-CISK behavior such as grid-scale noise
\end{quote}

\item \bibentry{KB08}\ \\
HYMACS -- hybrid mass flux convection scheme -- only convective updrafts and downdrafts are parameterised. Compesating subsidence transferred to the grid scale equations. Implemented in COSMO.\\
Model is $100\times 36\times35$ grid points with $\Delta x=7$km and $\Delta t=40$s. Convection scheme called every 400s

\item \bibentry{KGB07}\ \\
Convection scheme that alters density -- passes tendency to continuity eqn

\item \bibentry{GG05} \ \\
Prognostic equations for $w_u$ and $\sigma_u$ and lots of discussion of non-equilibrium but not clear how mass is actually transported. $\sigma_u$ is assumed uniform in the vertical.

\item \bibentry{Ger07}
\item \bibentry{AW13} \ \\
\includegraphics[scale=0.5,angle=90]{notes/AW13notes.jpg}

\item \bibentry{Par14} \ \\
no mass transport \\
UNICON simulates all dry–moist, forced–free, and shallow–deep convection within a single framework in a seamless, consistent, and unified way.

diagnoses the vertical profiles of the macrophysics (fractional area,
plume radius, and number density) as well as the microphysics (production and evaporation rates of convective
precipitation) and the dynamics (mass flux and vertical velocity) of multiple convective updraft and
downdraft plumes

successfully simulates, eg diurnal cycle and MJO

\item \bibentry{SMST04} \ \\
Looks like it includes useful closures. See also \url{http://adsabs.harvard.edu/abs/2014AGUFM.A13J3309T}

\item \bibentry{Yano14} \ \\ assume vanishing fractional area occupied by convection

\item \bibentry{YL14}
\item \bibentry{YRGB05}
\item \bibentry{BHL93}
\item \bibentry{BWF03}
\item \bibentry{Ger07}
\item \bibentry{GB99}
\item \bibentry{Kai04}
\item \bibentry{KF93}
\item \bibentry{KP12}
\item \bibentry{PC08}
\item \bibentry{PR98}
\item \bibentry{SP04}
\item \bibentry{WG10}
\item \bibentry{YP12}
%\item \bibentry{}
%\item \bibentry{}
%\item \bibentry{}
%\item \bibentry{}
%\item \bibentry{}
%\item \bibentry{}
%\item \bibentry{}
%\item \bibentry{}
%\item \bibentry{}
%\item \bibentry{}
%\item \bibentry{}
%\item \bibentry{}
%\item \bibentry{}
%\item \bibentry{}
%\item \bibentry{}
%\item \bibentry{}
%\item \bibentry{}
%\item \bibentry{}
%\item \bibentry{}
%\item \bibentry{}
%\item \bibentry{}

\section*{Multi-phase flows and Conditional Averaging}

\item \bibentry{KSP04} \ \\Uses conditional averaging
\item \bibentry{LB91} \ \\ Includes conditionally averaged momentum equation in advective form. Discusses interfacial momentum transfer

\item \bibentry{MTP99} \ \\ ``defnition and interpretation of average pressure in an incompressible disperse two-phase flow''

\item \bibentry{ZP97}  \ \\
%\includegraphics[width=\linewidth]{notes/ZP97.jpg}

\item \bibentry{Dopa77} \ \\
%\includegraphics[width=\linewidth]{notes/Dopa77.jpg}

\item \bibentry{ZP94a} \ \\
%\includegraphics[width=\linewidth]{notes/ZP94a.jpg}

\item \bibentry{ZP94b}

\item \bibentry{Wor03} \ \\
%\includegraphics[width=\linewidth]{notes/Wor03.jpg}

\item \bibentry{Wel05}  \ \\
%\includegraphics[width=\linewidth]{notes/openFoamCondAve.jpg}

%\item \bibentry{}

\end{itemize}

\nobibliography{numerics}

\end{document}
