%% LyX 2.0.6 created this file.  For more info, see http://www.lyx.org/.
%% Do not edit unless you really know what you are doing.
\documentclass[12pt,british,12pt, round,comma,sort&compress]{article}
\usepackage{mathptmx}
\renewcommand{\familydefault}{\rmdefault}
%\usepackage[T1]{fontenc}
%\usepackage[latin9]{inputenc}
\usepackage[a4paper]{geometry}
\usepackage{graphicx,amsmath}
\geometry{verbose,tmargin=2cm,bmargin=2cm,lmargin=2cm,rmargin=2cm}
\setcounter{secnumdepth}{5}
\setcounter{tocdepth}{5}
\setlength{\parskip}{\smallskipamount}
\setlength{\parindent}{0pt}

\makeatletter
%%%%%%%%%%%%%%%%%%%%%%%%%%%%%% User specified LaTeX commands.
\usepackage{color}
\newcommand{\nicefrac}[2]{\ensuremath ^{#1}\!\!/\!_{#2}}
\usepackage{fancybox}

\renewcommand{\floatpagefraction}{0.95}
\renewcommand{\textfraction}{0}
\renewcommand{\topfraction}{1}
\renewcommand{\bottomfraction}{1}

\usepackage{bibentry}
\bibliographystyle{abbrvnat}
\nobibliography{numerics}

\makeatother

\usepackage{babel}
\begin{document}

\title{Some Convection Parameterization Papers}
\author{Hilary Weller}
\maketitle

\subsection*{Convection parameterisation is a problem}
\begin{itemize}

\item 
Revcon proposal: \\
``in constructing a reliable and accurate parameterization underlie many current model deficiencies [IPCC, Ch. 8,10,12, K12, S10].''

``There are many differences in existing parameterizations but the
basic structure of many of our current operational methods has remained unchanged for decades
as highlighted in recent reviews [e.g. A04, H14, R13] and as highlighted by the RevCon project
subtitle.''

\item \bibentry{IPCC5_WG1} \ \\
7.2.3.3 ``improved simulations of the Madden--Julian Oscillation, tropical convectively coupled waves and mean rain-fall patterns in some models ... but usually at the expense of a degraded simulation of the mean state.''

Other improvements since AR4. 

\item \bibentry{KS12} \ \\
Does not mention convection

\item \bibentry{SLF+10} \ \\
``The accumulations
from the four general circulation models that employ more
traditional parameterization of moist physics tend to produce
too much precipitation over the tropical oceans compared
to observations.''

The version of CAM with a traditional parameterization ``produces too much light rainfall that occurs as soon as the boundary layer energizes''.

\item \bibentry{HPB+14} \ \\
``Atmospheric convection is arguably the biggest current
obstacle to the improvement of global weather and
climate prediction (e.g. Sherwood et al., 2014).''

``Short-term forecasts of
global convection remain poor, with biases that have not
been removed by increasing resolution in current oper-
ational forecast models (Lin et al., 2012)''

\item \bibentry{Ran13} \ \\
Work on conventional parameterizations has been about entrainment \\
Superparameterization have succeeded in simulation the MJO and the diurnal cycle of convection. Global high resolutions for resolved convection not feasible.

\item \bibentry{SAB+13} \ \\
``Physical processes not well resolved by climate models continue to limit confidence in detailed predictions of climate change. The representation of cloud and convection-related processes dominates the model spread in global climate sensitivity, and affects the simulation of important aspects of the present-day climate especially in the tropics.''

\item \bibentry{ipcc41} \ \\
Chapter 8: Problems with MJO and TCs due to convection parametrization

\item \bibentry{LCD+08} \ \\
``It is shown that the 4- and 1-km-gridlength models often give more realistic-looking precipitation fields because convection is represented
explicitly rather than parameterized. However, the 4-km model representation suffers from large convective cells and delayed initiation because the grid length is too long to correctly reproduce the convection explicitly.''

\item \bibentry{Ara04} \ \\
``The current trend in posing cumulus parameterization is away from deterministic diagnostic closures, including instantaneous adjustments, toward prognostic or nondeterministic closures, including relaxed
and/or triggered adjustments.'' \\
``Major  practical  and  conceptual  problems  in  the  conventional  approach  of  cumulus  parameterization, which
include  artificial  separations  of  processes  and  scales''

\end{itemize}

\subsection*{Others}
\begin{itemize}
\item \bibentry{GR90}
\item \bibentry{Tied89} \ \\
Moisture convergence for deep and mid-level convection \\
Surface evaportation for shallow convection
\item \bibentry{AS74} \ \\ See notes in \url{notes/AS74.pdf}
\item \bibentry{KF90}
\item \bibentry{Kai04}
\item \bibentry{LR01} \ \\
Demonstrates the equivalence between using a higher order closure and predicting $w^\prime w^\prime$ and $w^\prime w^\prime w^\prime$ with predicting mass flux and area fraction, $\sigma$. Transfer terms not clear

\item \bibentry{KH17} \ \\
\includegraphics[angle=90, width=\linewidth]{notes/KH17.jpg}

\item \bibentry{SBB+03} \ \\
In LES simulations -- prescribe the surfaces fluxes with observations. These are balanced by vertically integrating the prescribed large-scale forcing.\\
Spinup period of 2 hours where there is no resolved-scale turbulence to generate sufficient variability to create clouds. \\
One eddy turnover time approx 30 mins \\
Cloud cover 10-15\%\\
$\theta_v = \theta(1+0.61 q_v - q_\ell)$ \\
Two types of conditionally sampled files:\\
a) averaged over grid points with non-zero liquid water \\
b) non-zero liquid water and positively buoyant w.r.t. slab average \\
approx 50\% of the clouds are positively buoyant. \\
dowwdrafts play a minor role - appro\ 90\% of the cloud is in an updraft\\
The mass flux approximatin represents 80\%-90\% of the turbulent fluxes for the moist conserved variables $q_t$ and $\theta_\ell$ \\
Model setup -- the most important forcing is due to the subsidence - $\overline{w}$

\item \bibentry{RC15} \ \\
``algorithm is used to extract thousands of cloud thermals from a large-eddy simulation of deep and tropical maritime convection.'' \\
``Cloud thermals rise with a nearly constant speed determined by their buoyancy and the standard drag law with a drag coefficient of 0.6.'' \\
$F_{drag} = 0.5 c_d A w^2$.

\item \bibentry{HP98} \ \\
Convective trigger function is related to sub-grid-scale buoyancy which is related to turbulent heat flux

\end{itemize}

\subsection*{Recommended by Alan Grant on conditional averaging}
\begin{itemize}

\item \bibentry{SST07}\ \\
"(EDMF) is proposed for the dry
convective boundary layer. It is shown that the EDMF approach follows naturally from a decomposition of
the turbulent fluxes into 1) a part that includes strong organized updrafts, and 2) a remaining turbulent field.
At the basis of the EDMF approach is the concept that nonlocal subgrid transport due to the strong updrafts
is taken into account by the MF approach, while the remaining transport is taken into account by an ED closure."

"we define this strong updraft as a fixed fractional
area $a_u$ , say a few percent of the horizontal domain
under consideration, that contains the strongest upward
vertical velocities."

"convective mass flux can be defined as $M\equiv a_u (w_u - w)$. If we now
make use of the fact that $a_u << 1$.

\end{itemize}

\subsection*{Others on Convection}
\begin{itemize}



\item \bibentry{BFM16}\ \\
\begin{quote}
Observational studies show a strong correlation between large-scale wind convergence and precipitation. However, using this as a convective closure assumption to determine the total precipitation in a numerical model typically leads to deleterious wave-CISK behavior such as grid-scale noise
\end{quote}

\item \bibentry{KB08}\ \\
HYMACS -- hybrid mass flux convection scheme -- only convective updrafts and downdrafts are parameterised. Compesating subsidence transferred to the grid scale equations. Implemented in COSMO.\\
Model is $100\times 36\times35$ grid points with $\Delta x=7$km and $\Delta t=40$s. Convection scheme called every 400s

\item \bibentry{KGB07}\ \\
Convection scheme that alters density -- passes tendency to continuity eqn

\item \bibentry{GG05} \ \\
Prognostic equations for $w_u$ and $\sigma_u$ and lots of discussion of non-equilibrium but not clear how mass is actually transported. $\sigma_u$ is assumed uniform in the vertical.

\item \bibentry{Ger07}
\item \bibentry{AW13} \ \\
\includegraphics[scale=0.5,angle=90]{notes/AW13notes.jpg}

\item \bibentry{Par14} \ \\
no mass transport \\
UNICON simulates all dry-moist, forced-free, and shallow-deep convection within a single framework in a seamless, consistent, and unified way.

diagnoses the vertical profiles of the macrophysics (fractional area,
plume radius, and number density) as well as the microphysics (production and evaporation rates of convective
precipitation) and the dynamics (mass flux and vertical velocity) of multiple convective updraft and
downdraft plumes

successfully simulates, eg diurnal cycle and MJO

\item \bibentry{SMST04} \ \\
Looks like it includes useful closures. See also \url{http://adsabs.harvard.edu/abs/2014AGUFM.A13J3309T}

\item \bibentry{Yano14} \ \\ assume vanishing fractional area occupied by convection

\item \bibentry{YL14}
\item \bibentry{YRGB05}
\item \bibentry{BHL93}
\item \bibentry{BWF03}
\item \bibentry{Ger07}
\item \bibentry{GB99}
\item \bibentry{Kai04}
\item \bibentry{KF93}
\item \bibentry{KP12}
\item \bibentry{PC08}
\item \bibentry{PR98}
\item \bibentry{SP04}
\item \bibentry{WG10}
\item \bibentry{YP12}

\section*{Entrainment and Detrainment}

\item\bibentry{DBF+13}\ \\
Recommended by reviewer:\\
``the mass diffusion and the divergence regulators are related to the turbulent and dynamical entrainment and detrainment) respectively''

``lateral entrainment is the dominant mixing mechanism in comparison with the cloud-top entrainment in shallow cumulus convection''

``Observational evidence (e.g. Jonas, 1990; Rodts et al., 2003) as well as an
LES study (Heus and Jonker, 2008) reveal the existence of a descending shell with properties significantly different from the average environment (section 3.8).''

``extensive variety of parametrisations has been developed for $\varepsilon$ and $\delta$ without a sign of any convergence''

\item\bibentry{DSJD12}\ \\
Recommended by reviewer:\\
``the momentum drag regulator is related to the pressure perturbation''

From the abstract: \\
``an analysis is presented of the full vertical budget equation for shallow cumulus clouds obtained from large eddy simulations of three different Global Energy and Water Cycle Experiment (GEWEX) Cloud System Study (GCSS) intercomparison cases. It is found that the pressure gradient term is the dominant sink term in the vertical velocity budget, whereas the entrainment term only gives a small contribution. This result is at odds with the parameterized vertical velocity equation in the literature as it employs the entrainment
term as the major sink term''

``most moist convection schemes use a vertical velocity
equation in which the in-cloud vertical velocity $w_c$ is controlled by the in-cloud buoyancy excess with respect to the
environment of the cloud $B_c$ , and a sink term that is often
taken proportionally to the fractional entrainment rate $\epsilon$:
\[
\frac{1}{2}\frac{\partial w_c^2}{\partial z} = aB_c - -b\epsilon w_c^2
\]
The effects of nonhydrostatic
pressure perturbations and subplume fluctuations are
believed to be taken into account by a reduction of the
buoyancy term ($a<1$) and by a lateral entrainment term
multiplied by a proportionality factor $b$.

\item \bibentry{LR01b} \ \\
Gives the entrainment and detrainment rates for the \cite{LR01}. 

\item \bibentry{SW69} \ \\
Gives PDE for dw/dt as a balance between buoyancy and drag and entrainment. Drag is
$\frac{3}{8}C_D \frac{w^2}{R}$.


\item \bibentry{TEB19} \ \\

``Mass flux schemes are known to be particlularly sensitive to the parameterization of entrainment and detrainment (e.g. Romps 2016, and references therein). EDMF schemes for the convective boundary layer have used a variety of forms for the fractional entrainment per unit depth $\varepsilon$ , including a constant value (Angevine 2005), a specified function of normalized depth (Soares et al. 2004; Siebesma et al. 2007), a rate inversely proportional to updraft speed (Neggers et al. 2009), and a rate inversely proportional to a
diagnosed turbulence length scale (Witek et al. 2011b).

Many EDMF schemes prescribe the vertical profile of mass flux or of updraft volume fraction within the boundary layer or subcloud layer. When this is done the vertical profile of detrainment is implied by the updraft mass budget and so does not need to be explicitly parameterized. The scheme of Angevine (2005), Angevine et al. (2010) is one example that does explicitly parameterize detrainment; the fractional detrainment rate per unit depth $\delta$ is a simple profile of $z$ that, in the absence of condensation, peaks at the
boundary layer top.''

``From Cheinet (2003):
\begin{equation*}
\varepsilon = \max\big(\frac{w_*}{z_* w_2}, \frac{c_\varepsilon}{z}\big)
\end{equation*}
with $c_\varepsilon = 0.4$.''

For the dry convective BL, need to ``prevent the build-up of a large volume fraction of fluid 2 at the boundary layer top''. Derived from Siebesma et al (2007):
\[
\delta = \frac{1}{2}\frac{1}{z_* - z}
\]

\item \bibentry{AK67} \ \\

\item \bibentry{HC51} \ \\
Entrainment is a necessary dynamic consequence of vertical acceleration


\item \bibentry{LJ1x} \ \\
DNS simulations. Entrainment is only weakly related to Reynolds number. More closely related to 1/r, following theory.

entrainment rate = e/r (units 1/distance)\\
e is 0.36, 0.47, 0.75, 0.5 in different experiments\\
Entrainment leads to ``mixing drag'' \\
Thermals increase in radius as they rise

%\item \bibentry{}
%\item \bibentry{}
%\item \bibentry{}
%\item \bibentry{}
%\item \bibentry{}
%\item \bibentry{}
%\item \bibentry{}
%\item \bibentry{}
%\item \bibentry{}
%\item \bibentry{}
%\item \bibentry{}
%\item \bibentry{}
%\item \bibentry{}
%\item \bibentry{}
%\item \bibentry{}
%\item \bibentry{}
%\item \bibentry{}
%\item \bibentry{}
%\item \bibentry{}

\section*{Multi-phase flows and Conditional Averaging}

\item \bibentry{GBB+07} \ \\
``interfacial transfer terms, which
are taken into account only if they act as return-to-
equilibrium terms and are easy to derive.'' \\
single pressure \\
First order upwind for advecting volume fraction


\item \bibentry{KSP04} \ \\Uses conditional averaging

\item \bibentry{LB91} \ \\ Includes conditionally averaged momentum equation in advective form. Discusses interfacial momentum transfer\\
Uses drag on bubbles

\item \bibentry{MTP99} \ \\ ``definition and interpretation of average pressure in an incompressible disperse two-phase flow'' \\
``The definition and interpretation of average pressure in an incompressible disperse two-phase  ̄ow are
ambiguous and have been the object of debate in the literature. For example, the physical meaning of
definitions involving an internal `pressure' inside rigid particles is unclear.''

\item \bibentry{ZP97}  \ \\
%\includegraphics[width=\linewidth]{notes/ZP97.jpg}

``We consider an ensemble of macroscopically identical suspensions of N spherical particles in a
fluid continuous phase.''

All theory, no numerical solutions

\item \bibentry{Dopa77} \ \\
%\includegraphics[width=\linewidth]{notes/Dopa77.jpg}

\item \bibentry{ZP94a} \ \\
%\includegraphics[width=\linewidth]{notes/ZP94a.jpg}

\item \bibentry{ZP94b}

\item \bibentry{Wor03} \ \\
%\includegraphics[width=\linewidth]{notes/Wor03.jpg}

\item \bibentry{Wel05}  \ \\
%\includegraphics[width=\linewidth]{notes/openFoamCondAve.jpg}

\item \bibentry{HK86a}
\item \bibentry{HK86b}

\item \bibentry{HK84} \ \\
``The standard therory of ideal single-pressure multiphase fluid dynamics, which is know to be ill-posed, is regularized via the hamiltonian formalism by extending the noncanical Poisson brackets for the standard single-pressure equations to the case of multiple pressures. This formalism is used to find Lyapunov stability conditions for the regularized system''

Single pressure system has complex speed of sound - not hyperbolic. Two pressures regularises it.

\end{itemize}

\nobibliography{numerics}

\end{document}
