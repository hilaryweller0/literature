%% LyX 2.0.6 created this file.  For more info, see http://www.lyx.org/.
%% Do not edit unless you really know what you are doing.
\documentclass[12pt,british,12pt, round,comma,sort&compress]{article}
\usepackage{mathptmx}
\renewcommand{\familydefault}{\rmdefault}
%\usepackage[T1]{fontenc}
%\usepackage[latin9]{inputenc}
\usepackage[a4paper]{geometry}
\usepackage{graphicx}
\geometry{verbose,tmargin=2cm,bmargin=2cm,lmargin=2cm,rmargin=2cm}
\setcounter{secnumdepth}{5}
\setcounter{tocdepth}{5}
\setlength{\parskip}{\smallskipamount}
\setlength{\parindent}{0pt}

\makeatletter
%%%%%%%%%%%%%%%%%%%%%%%%%%%%%% User specified LaTeX commands.
\usepackage{color}
\newcommand{\nicefrac}[2]{\ensuremath ^{#1}\!\!/\!_{#2}}
\usepackage { fancybox}

\renewcommand{\floatpagefraction}{0.95}
\renewcommand{\textfraction}{0}
\renewcommand{\topfraction}{1}
\renewcommand{\bottomfraction}{1}

\usepackage{bibentry}
\bibliographystyle{abbrvnat}
\nobibliography{numerics}

\makeatother

\usepackage{babel,amsmath}
\begin{document}

\title{Physics Dynamics Coupling}
\author{Hilary Weller}
\maketitle

\begin{itemize}

\item \bibentry{CLZ98} \ \\
\includegraphics[width=\textwidth]{notes/CLZ98.jpg}
\item \bibentry{Wil02} \ \\
\includegraphics[width=\textwidth]{notes/Wil02.jpg}

\item \bibentry{CS03} \ \\
``Implicit ... are not practical because of the nonlinear switching present in most parametrization schemes''
\item \bibentry{BBK+04}


\item \bibentry{GWR+16}
\item \bibentry{NRC+10}

\item \bibentry{DWS04}
\item \bibentry{DWS05}
\item \bibentry{DWS06}

%Bacmeister, J., P. Lauritzen, A. Dai, and J. Truesdale (2012),
%Assessing possible dynamical effects of condensate in high resolution climate models, Geophys. Res. Lett., 39, L04,806.

%Beare, R. J., and M. J. P. Cullen (2015), Convergence to Ekman-
%balanced states: a test of physics-dynamics coupling, Quar-
%terly Journal of the Royal Meteorological Society, submitted.


%Staniforth, A., N. Wood, and J. Cˆot ́e (2002a), Analysis of the
%numerics of physics-dynamics coupling, Quarterly Journal of
%the Royal Meteorological Society, 128 (586), 2779–2799, doi:
%10.1256/qj.02.25.

%Staniforth, A., N. Wood, and J. Cˆot ́e (2002b), A simple com-
%parison of four physics dynamics coupling schemes, Monthly
%Weather Review, 130 (12), 3129–3135, doi:10.1175/1520-
%0493(2002)130¡3129:ASCOFP¿2.0.CO;2.

%Wedi, N. (1999), The numerical coupling of the physical param-
%eterizations to the ”dynamical” equations in a forecast model,
%Technical Memorandum 274, ECMWF, Reading, U.K.


\item \bibentry{WBK+08} \ \\
PC2 -- prognostic cloud fraction and prognostic condensate scheme \\
-- $q_v$ and $q_\ell$ and $q_i$ and $T$ are prognostic


\item \bibentry{KR03} \ \\
``The prognostic thermodynamical variables of the
model are the liquid water/ice moist static energy, total
nonprecipitating water (vapor 1 cloud water 1 cloud
ice), and total precipitating water (rain 1 snow 1 graupel). The liquid water/ice moist static energy is, by definition, conserved during the moist adiabatic processes
including the freezing/melting of precipitation. The
cloud condensate (cloud water 1 cloud ice) is diagnosed
using the so-called ''all-or-nothing`` approach, so that
no supersaturation of water vapor is allowed''

$h_L$ is liquid/ice water static energy\\
$h_L = c_p T + g z + L_c (q_c + q_r) - L_s(q_i + q_s + q_g)$

\item \bibentry{LEB14} \ \\
UM BL scheme has prognostic variables\\
$\theta_\ell = T_L + \frac{g}{c_p}z
= T - \frac{L}{c_p}q_\ell - \frac{L_s}{c_p}q_f + \frac{g}{c_p}z$ \\
$q_t = q_v + q_\ell + q_f$

\item \bibentry{Smi90}\ \\
Prognostic variable, total cloud water content $q_C$ which includes liquid and ice, dependent on the temperature. Also temperature and specific humidity. \\
Turbulence scheme uses cloud conserved variables -- total water content and liquid--frozen water temperature:\\
$T_L = T - (L_C/c_p)q_L - (L_C+L_F)$

\item \bibentry{YM79} \ \\
Prognostic variables of total water mixing ratio and wet bulb potential temperature

\item \bibentry{RH83b} \ \\
They cite \cite{YA79} for formula for condenstation rate

\item \bibentry{YA79}\ \\
They cite \cite{Asai65} for formula for condenstation rate

\item \bibentry{Asai65}\ \\
The condensation rate is limited by the amount that can condense in one time-step given the temperature change.

\item \bibentry{DWD06}\ \\
boundary layer coupling can produce two-time-step ocillations which using preditor-corrector approach (Kalnay and Kanmitsu)\\
New scheme -- unconditional stability and 2nd order accuracy for linear or mildly nonlinear problems. 

\item \bibentry{KK88}\ \\
Vertical diffusion acts as a strongly non-linear damping terms. This can lead to unreaslistic oscillations in time. Dependent variable (eg temperature) needs to be treated implicitly but for simplicity, coefficients are treated explicitly (with predictor-corrector).

\item \bibentry{WDS07}\ \\
Makes assumptions about the non-linearity of the diffusion coefficient when updating the diffusion coefficient using a predictor-corrector approach and an implicit scheme for the linearised diffusion equation.

\item \bibentry{BC16}\ \\
Asymptotic methods to validate boundary layer -- dynamics coupling and expose the weakness of implicit methods that do not treat the non-linear diffusion coefficient accurately. 

\item \bibentry{Thu17}\ \\
``The thermodynamics of moist processes is complicated, and in typical atmospheric models
numerous approximations are made. However, they are not always made in a self-consistent
way, which could lead to spurious sources or sinks of energy and entropy. One way to
ensure self-consistency is to derive all thermodynamic quantities from a thermodynamic
potential such as the Gibbs function. Approximations may be made to the Gibbs function;
these approximations are inherited by all derived quantities in a way that guarantees
self-consistency.''

Entropy, $\eta = C_p \ln \theta + \text{const}$

The Gibbs function is defined by\\
$g(p,R,q) = e + \alpha p - \eta T$\\
where $e$ is specific internal energy and $\alpha = 1/\rho$ is specific volume. Eqns of state to use in a model are\\
$1/\rho - g_p(p,T,q) = 0$\\
$\eta + g_T(p,T,q) = 0$\\

\item \bibentry{SWC02}
%\item \bibentry{}
%\item \bibentry{}
%\item \bibentry{}
%\item \bibentry{}

\end{itemize}

\nobibliography{numerics}

\end{document}
