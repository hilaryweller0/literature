\documentclass[12pt]{article}
\usepackage[a4paper, tmargin=2cm,bmargin=2cm,lmargin=2cm,rmargin=2cm]{geometry}
\usepackage{graphicx}
\setlength{\parskip}{\smallskipamount}
\setlength{\parindent}{0pt}

\usepackage{bibentry}
\bibliographystyle{abbrvnat}
\nobibliography{numerics}

\begin{document}

\title{Literature review on Time-Stepping Schemes}
\author{Hilary Weller}
\maketitle

\begin{itemize}
\item \bibentry{VSR+19} \ \\
``The standard strategy of using fully implicit methods for integrating numerically stiff equations can be a computational burden in operational models (Evans et al., 2017; Lott et al., 2015)''
\item \bibentry{LM04} \ \\
Uses deferred correction rather than Strang splitting to achieve high order and avoids doing chemistry on a state that is out of balance
\item \bibentry{WLW13}
\item \bibentry{PR05}
\item \bibentry{ARS97}
\item \bibentry{DB12}
\item \bibentry{WLW13}
\item \bibentry{LWW14}
\item \bibentry{IGG18}
\item \bibentry{GS98} \ \\ \includegraphics[width=\linewidth]{notes/GS98.jpg}
\item \bibentry{GST01} \ \\ \includegraphics[width=\linewidth]{notes/GST01.jpg}
\item \bibentry{FS08} \ \\ Defines SDIRK \\
Strongly stable SDIRK methods have the order barrier $p\le 4$.
\item \bibentry{KMG09} \ \\
``We consider methods up to order six (the maximal order of SSP Runge–Kutta methods) and up to eleven stages. The
numerically optimal methods found are all diagonally implicit, leading us to conjecture that optimal implicit SSP Runge-Kutta
methods are diagonally implicit. These methods allow a larger SSP timestep, compared to explicit methods of the same order and
number of stages.'' \\
``Result 1. Any Runge-Kutta method of order $p > 1$ has a finite radius of absolute monotonicity'' \\
``Result 7. An SDIRK method with positive radius of absolute monotonicity $R(\mathbf{K}) > 0$ must have order $p \le 4$.

\item \bibentry{BD15} \ \\ ``Two hybrid variants of TR-BDF2 are proposed, that reduce the formal order of accuracy and maximize the absolute monotonicity radius''
\end{itemize}

\nobibliography{numerics}

\end{document}
