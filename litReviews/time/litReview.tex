\documentclass[12pt]{article}
\usepackage[a4paper, tmargin=2cm,bmargin=2cm,lmargin=2cm,rmargin=2cm]{geometry}
\usepackage{graphicx}
\setlength{\parskip}{\smallskipamount}
\setlength{\parindent}{0pt}

\usepackage{bibentry}
\bibliographystyle{abbrvnat}
\nobibliography{numerics}

\begin{document}

\title{Literature review on Time-Stepping Schemes}
\author{Hilary Weller}
\maketitle

\begin{itemize}
\item \bibentry{VSR+19} \ \\
``The standard strategy of using fully implicit methods for integrating numerically stiff equations can be a computational burden in operational models (Evans et al., 2017; Lott et al., 2015)''
\item \bibentry{LM04} \ \\
Uses deferred correction rather than Strang splitting to achieve high order and avoids doing chemistry on a state that is out of balance
\item \bibentry{WLW13}
\item \bibentry{PR05}
\item \bibentry{ARS97}
\item \bibentry{DB12}
\item \bibentry{WLW13}
\item \bibentry{LWW14}
\item \bibentry{IGG18}
\item \bibentry{GS98} \ \\ \includegraphics[width=\linewidth]{notes/GS98.jpg}
\item \bibentry{GST01} \ \\ \includegraphics[width=\linewidth]{notes/GST01.jpg}
\item \bibentry{FS08} \ \\ Defines SDIRK \\
Strongly stable SDIRK methods have the order barrier $p\le 4$.
\item \bibentry{KMG09} \ \\
``We consider methods up to order six (the maximal order of SSP Runge–Kutta methods) and up to eleven stages. The
numerically optimal methods found are all diagonally implicit, leading us to conjecture that optimal implicit SSP Runge-Kutta
methods are diagonally implicit. These methods allow a larger SSP timestep, compared to explicit methods of the same order and
number of stages.'' \\
``Result 1. Any Runge-Kutta method of order $p > 1$ has a finite radius of absolute monotonicity'' \\
``Result 7. An SDIRK method with positive radius of absolute monotonicity $R(\mathbf{K}) > 0$ must have order $p \le 4$.

\item \bibentry{BD15} \ \\ ``Two hybrid variants of TR-BDF2 are proposed, that reduce the formal order of accuracy and maximize the absolute monotonicity radius''
\end{itemize}

\section*{Local Time Stepping}

\begin{itemize}
\item \bibentry{Bald19} \ \\ 
Implemented in COSMO. Local time stepping maintains numerical reproducibility and exact mass conservation ``together with a slight improvement in
efficiency.''

\item \bibentry{DKT07} \ \\ 
Local time stepping needed for high flows in small elements. \\
``Please note,
however, that the efficient parallel execution of a LTS scheme can become a very challenging task. Since each element performs its update
only when (15) is fulfilled locally, the element updates are in general completely asynchronous which induces a very irregular load balancing
on the different processors. Therefore, for the parallel implementation of our code, we first group elements of zones with similar material
properties and element size together and partition each of these zones separately into a number of partitions that is equal to the total number
of processors. For this purpose we are using the free METIS software package introduced by Karypis & Kumar (1998). Then, the resulting
partitions of the different zones are merged together and thus provide the final domain decomposition. Note that the final partition in general
contains non-connected subdomains. However, since the ADER-DG approach provides a one-step scheme, the total communication overhead
induced by these non-contiguous partitions is quite small compared to the achieved benefits from the improved load balancing. Note that just
using METIS on the whole domain with large weights on elements with small time steps can not resolve the problem since the processor
load may remain ill-balanced due to the asynchronous element updates. As a final remark on parallelization we would like to note that in our
implementation the MPI communication is always performed after each cycle.''

\item \bibentry{DTKW13} \ \\ 
Local time stepping motivated by small time steps due to high flow rates in small regions and on small elements. Parallel version. \\
``First, there is the question of load balancing. The amount of
work per element depends on the local timestep. We have at-
tempted to address this in METIS by weighting each node by a fac-
tor which depends on the local timesteps associated with elements
attached to the node. This factor is determined by the number of
sub-cycling steps required for the element with the smallest time-
step to go from time t n to time t nþ1 . The local timestep may also
change during the course of the simulation, therefore in reality
the load should be dynamical ly re-balanced during the simulation.''

\item \bibentry{} \ \\ 
\end{itemize}

\nobibliography{numerics}

\end{document}
