%% LyX 2.0.6 created this file.  For more info, see http://www.lyx.org/.
%% Do not edit unless you really know what you are doing.
\documentclass[12pt,british,12pt, round,comma,sort&compress]{article}
\usepackage{mathptmx}
\renewcommand{\familydefault}{\rmdefault}
%\usepackage[T1]{fontenc}
%\usepackage[latin9]{inputenc}
\usepackage[a4paper]{geometry}
\usepackage{graphicx}
\geometry{verbose,tmargin=2cm,bmargin=2cm,lmargin=2cm,rmargin=2cm}
\setcounter{secnumdepth}{5}
\setcounter{tocdepth}{5}
\setlength{\parskip}{\smallskipamount}
\setlength{\parindent}{0pt}

\makeatletter
%%%%%%%%%%%%%%%%%%%%%%%%%%%%%% User specified LaTeX commands.
\usepackage{color}
\newcommand{\nicefrac}[2]{\ensuremath ^{#1}\!\!/\!_{#2}}
\usepackage { fancybox}

\renewcommand{\floatpagefraction}{0.95}
\renewcommand{\textfraction}{0}
\renewcommand{\topfraction}{1}
\renewcommand{\bottomfraction}{1}

\usepackage{bibentry}
\bibliographystyle{abbrvnat}
\nobibliography{numerics}

\makeatother

\usepackage{babel}
\begin{document}

\title{Turbulence}
\author{Hilary Weller}
\maketitle

\begin{itemize}

\item \bibentry{CHB+06
\item \bibentry{Duy88}
\item \bibentry{Wyn75} \ \\
second moments -- `higher-order-closure'. \\
Transport equations fo turbulence co-variances, $\overline{u_i u_k}$, $\overline{\theta^2}$ and $\overline{\theta u_i}$

\end{itemize}

\section*{Convection Related}

\begin{itemize}
\item \bibentry{SRLV19} \ \\
From the abstract: \\
``The ability of three parametrizations—the traditional K-gradient model
of Cuxart et al. (CBR), the Smagorinsky formulation, and the Moeng approach
(MOENG) based on the horizontal gradients of the resolved variables -- to reproduce
the thermodynamical and dynamical fluxes was assessed in an offline configuration via a comparison with the reference fields. MOENG was the most appropriate
scheme to represent the vertical and horizontal heat fluxes at all grid spacings in the
clouds and their environment over the entire cloud life cycle, including the appropriate representation of countergradient areas.''

CBW uses TKE as a prognostic then has algebraic expressions for turbulent momentum, heat and moisture fluxes as a function of TKE and mean gradients. 

Smagorinsky just uses mean gradients

``mixing length of Bougeault and Lacarrere (1989),
which is equivalent to the distance a parcel of air with an initial kinetic energy $e(z)$ given by its altitude can travel before
it is stopped by buoyancy effects.'' An algebraic equation.

``Moeng (2014) parametrized the turbulent fluxes based on
products of the horizontal gradients of the resolved variables.''

\end{itemize}

\section*{Reynolds-Averaged Modelling}

\begin{itemize}
\item \url{https://www.sharcnet.ca/Software/Ansys/16.2.3/en-us/help/cfx_thry/i1302321.html}
\item \url{https://www.hindawi.com/journals/isrn/2012/428671/}
\item \url{www3.nd.edu/~gtryggva/CFD-Course/2011-Lecture-35.pdf}
\item \url{http://www.cham.co.uk/phoenics/d\_polis/d\_lecs/general/turb.htm}
\item \url{https://en.wikipedia.org/wiki/K-epsilon\_turbulence\_model}
\item \url{cfd.mace.manchester.ac.uk/twiki/pub/Main/TimCraftNotes\_All.../flmsc-rathermal.pdf}
\item \bibentry{YWL09}
\item \bibentry{GL78}
\item \bibentry{Gas18} \ \\
Gives a closure for buoyancy production in TKE equation

\item \bibentry{HvH91} \ \\
Cited in OpenFOAM for buoyantKEpsilon turbulence model

%\item \bibentry{}
%\item \bibentry{}
%\item \bibentry{}

\end{itemize}

\section*{Boundary Layer Parameterization}

\begin{itemize}
\item \bibentry{LBB+00}
\item \bibentry{SvH06}
\item \bibentry{HP96}
\item \bibentry{KK88}
%\item \bibentry{CS03}
%\item \bibentry{DWS06}
%\item \bibentry{}

\end{itemize}

\nobibliography{numerics}

\end{document}
