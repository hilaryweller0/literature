%% LyX 2.0.6 created this file.  For more info, see http://www.lyx.org/.
%% Do not edit unless you really know what you are doing.
\documentclass[12pt,british,12pt, round,comma,sort&compress]{article}
\usepackage{mathptmx}
\renewcommand{\familydefault}{\rmdefault}
%\usepackage[T1]{fontenc}
%\usepackage[latin9]{inputenc}
\usepackage[a4paper]{geometry}
\usepackage{graphicx}
\geometry{verbose,tmargin=2cm,bmargin=2cm,lmargin=2cm,rmargin=2cm}
\setcounter{secnumdepth}{5}
\setcounter{tocdepth}{5}
\setlength{\parskip}{\smallskipamount}
\setlength{\parindent}{0pt}
\usepackage{mathrsfs}

\makeatletter
%%%%%%%%%%%%%%%%%%%%%%%%%%%%%% User specified LaTeX commands.
\usepackage{color}
\newcommand{\nicefrac}[2]{\ensuremath ^{#1}\!\!/\!_{#2}}
\usepackage { fancybox}

\renewcommand{\floatpagefraction}{0.95}
\renewcommand{\textfraction}{0}
\renewcommand{\topfraction}{1}
\renewcommand{\bottomfraction}{1}

\usepackage{bibentry}
\bibliographystyle{abbrvnat}
\nobibliography{numerics}

\makeatother

\usepackage{babel}
\begin{document}

\title{Turbulence}
\author{Hilary Weller}
\maketitle

\begin{itemize}

\item \bibentry{ZWX20} \ \\
Includes a buoyancy production term for $\varepsilon$ equation.

\item \bibentry{Wyn75} \ \\
second moments -- `higher-order-closure'. \\
Transport equations for turbulence co-variances, $\overline{u_i u_k}$, $\overline{\theta^2}$ and $\overline{\theta u_i}$

\end{itemize}

\section*{Channel Flow}

\begin{itemize}

\item \bibentry{Hedl14} \ \\
Early DNS results for channel
flow were presented by Kim, Moin and Moser [3], and a continuation of this
work, which includes higher Re-numbers, where presented by Moser, Kim
and Mansour [4].

\item \bibentry{MKM99}

\item \bibentry{KMM87}


\item \bibentry{LM15} \ \\
``to investigate the differences between channel flow turbulence and boundary layer turbulence, a simulation at $Re_\tau=2000$ of a zero-pressure-gradient boundary layer was performed by Sillero, Jim\'enez and Moser (2013).''\\
``The flow is driven by a uniform pressure gradient, which varies in time to ensure that the mass flux through the channel remains constant.''\\
Data at \url{http://turbulence.ices.utexas.edu}\\

\item \bibentry{KABA05} \ \\
``outer layer is very little affected by the roughness''\\
$Re_\tau$ is Reynolds number based on friction velocity.\\
DNS $Re_\tau\le 400$.\\
Lab, $Re_\tau\ge 600$.\\

\item \bibentry{LODA06} \ \\
``The roughness elements consist of transverse square rods of size $k$, placed
on one wall of the channel only.''

\item \bibentry{JH15} \ \\
Compare 7 turbulence models with DNS.\\
RANS simulations with Fluent are 2D, resolving roughness elements.\\
RANS simulations are of a channel with just one rough and one smooth wall\\
Prefer the ``parameter extension method (PEM) was recently introduced by
Jin and Herwig''.\\
Have emailed about data availability.

\item \bibentry{ID07} \ \\
Rough on one side only\\
$Re_\tau=460$ \\
Compared with 2D RANS. \\
Digitise at \url{https://apps.automeris.io/wpd/}

\item \bibentry{BS11} \ \\
DNS but not resolving the roughness (uses a parameter in NS instead)\\
Symmetric channel. Graphs only of velocity profiles.

\item \bibentry{TBB00} \ \\ Not available to UoR

\item \bibentry{BS12} \ \\
``The roughness term is in all cases applied symmetrically to the upper and lower
parts of the channel, with the shape function being symmetric to the centreline of the channel.''\\
$\Delta U^+(k_s^+) = \kappa^{-1} \ln (0.3 k_s^+ + 1)$

\item \bibentry{CMK77}

\item \bibentry{BTS17} \ \\
Mean velocity data and pdfs of components of TKE.\\
Simulations up to $Re_\tau=720$\\
Velocity and velocity defects for rough walls plotted.

Flow driven by mean streamwise pressure gradient, $\Pi$. \\
mean channel half-height $\delta$.\\
All velocities non-dimensionalised by friction velocity $u_\tau=\sqrt{-\frac{\delta}{\rho}\Pi}$.\\
Both surfaces were scaled to the same mean peak-to-valley height $S_{z,5\times 5}=0.167$ (non-dimensionalised by $\delta$).\\
$S_{z,5\times 5}=0.167$ equivalent sand-grain roughness\\

\item \bibentry{Aupo15} \ \\
``equations in the wall region are solved, neglecting pressure gradient''\\
Equations made dimensionless with friction velocity and viscosity.\\
250 grid points from $y^+=0.01$ to $y^+=38,000$ with expansion ratio of 1.05.\\
For a smooth wall, $k^+$ is null at the wall and $\omega^+=6/\beta y^{+2}$ is imposed at the first grid point.\\
For a rough wall, finite values of $k^+$ and $\omega^+$ are imposed at the wall, according to the considered roughness correction.\\
The boundary conditions in the logarithmic region are the equilibrium logarithmic region values $k^+=1/\sqrt{\beta^*}$, $\omega^+=1/\sqrt{\beta^*}\kappa y^+$.\\
Only shows the velocity shift between smooth and rough. 

\item \bibentry{Thak17} \ \\
``2.3 DNS of turbulent channel flow''\\
\includegraphics[width=\textwidth]{/home/hilary/papers/PhDs/Thakkar17Data/Fig2p4.png}\\
Friction Reynolds number: $Re_\tau =\frac{u_\tau \delta}{\nu}$\\
$\Delta z_{\min}^+ <1$ and $\Delta z_{\max}^+ \le 5$.\\
NS equations non-dimensionalised by $\delta$ and $u_\tau$.\\
Channel flow driven by constant mean streamwise pressure gradient\\
$u_\tau^2 = -\frac{\delta}{\rho}\frac{dP}{dx} = 1$\\
$f_{emb}$ is an additional forcing function term in the non-dimensionalised momm eqn to account for roughness.\\
Roughness function, or roughness effect (or deficit), $\Delta U^+$ is the difference between $U^+$ for a rough and smooth surface.

Smooth-wall simulations\\
\includegraphics[width=\textwidth]{/home/hilary/papers/PhDs/Thakkar17Data/tab4p1.png}

\end{itemize}

\section*{Convection Related}

\begin{itemize}
\item \bibentry{SRLV19} \ \\
From the abstract: \\
``The ability of three parametrizations—the traditional K-gradient model
of Cuxart et al. (CBR), the Smagorinsky formulation, and the Moeng approach
(MOENG) based on the horizontal gradients of the resolved variables -- to reproduce
the thermodynamical and dynamical fluxes was assessed in an offline configuration via a comparison with the reference fields. MOENG was the most appropriate
scheme to represent the vertical and horizontal heat fluxes at all grid spacings in the
clouds and their environment over the entire cloud life cycle, including the appropriate representation of countergradient areas.''

CBW uses TKE as a prognostic then has algebraic expressions for turbulent momentum, heat and moisture fluxes as a function of TKE and mean gradients. 

Smagorinsky just uses mean gradients

``mixing length of Bougeault and Lacarrere (1989),
which is equivalent to the distance a parcel of air with an initial kinetic energy $e(z)$ given by its altitude can travel before
it is stopped by buoyancy effects.'' An algebraic equation.

``Moeng (2014) parametrized the turbulent fluxes based on
products of the horizontal gradients of the resolved variables.''


\item \bibentry{BSS22}

\item \bibentry{CD93} \ \\
\includegraphics[width=0.6\linewidth]{notes/CD93.png}

\item \bibentry{SST07} \ \\
\includegraphics[width=0.6\linewidth]{notes/SST07a.pdf}
\includegraphics[width=0.6\linewidth]{notes/SST07b.pdf}
\includegraphics[width=0.6\linewidth]{notes/SST07c.pdf}


\item \bibentry{SC95}

\item \bibentry{TEB19} \ \\
\includegraphics[width=0.6\linewidth]{notes/TEB19.pdf}

\end{itemize}

\section*{Reynolds-Averaged Modelling}

\begin{itemize}
\item \bibentry{LS74}

\item Tennekes and Lumley, 1972 

\item \url{https://www.sharcnet.ca/Software/Ansys/16.2.3/en-us/help/cfx\_thry/i1302321.html}
\item \url{https://www.hindawi.com/journals/isrn/2012/428671/}
\item \url{www3.nd.edu/\~{ }gtryggva/CFD-Course/2011-Lecture-35.pdf}
\item \url{http://www.cham.co.uk/phoenics/d\_polis/d\_lecs/general/turb.htm}
\item \url{https://en.wikipedia.org/wiki/K-epsilon\_turbulence\_model}
\item \url{cfd.mace.manchester.ac.uk/twiki/pub/Main/TimCraftNotes\_All.../flmsc-rathermal.pdf}
\item \bibentry{YWL09}
\item \bibentry{GL78}
\item \bibentry{Gas18} \ \\
Gives a closure for buoyancy production in TKE equation

\item \bibentry{HvH91} \ \\
Cited in OpenFOAM for buoyantKEpsilon turbulence model

\item \bibentry{Reyn76} \ \\
Review of zero, one and two equations models, Reynods stress modelling and LES.\\

The momentum equation \\
\includegraphics[width=\linewidth]{notes/Rey76_Ueqn.png}

The TKE equation \\
\includegraphics[width=\linewidth]{notes/Rey76_TKEeqn.png}

Closing the TKE equation \\
\includegraphics[width=\linewidth]{notes/Rey76_closingTKE.png}

Dissipation equation \\
\includegraphics[width=\linewidth]{notes/Rey76_dissiEqn.png}


\item \bibentry{Wil91} \ \\
k-omega is older than k-epsilon and has advantages\\
More accurate in adverse pressure gradients, through the sub-layer and in low Reynolds number. Boussinesq

\item \bibentry{Mironov09}

\item \bibentry{MGLZ99} \ \\
``Abdella and McFarlane (1997, henceforth AM97)
have proposed a second-order turbulence closure
scheme for the planetary boundary layer. The scheme
contains a prognostic equation for the turbulence kinetic
energy and algebraic expressions for the other second-
order moments. The expressions for the potential temperature flux and the temperature variance incorporate
nonlocal eddy diffusivity and countergradient terms.
These expressions are derived through the use of an
advanced parameterization of the third-order moments
based on convective mass-flux arguments. Remarkably,
the AM97 parameterizations for the flux of potential
temperature flux, $\overline{w^{\prime 2}\theta^\prime}$, and the flux of potential temperature variance, $\overline{w^\prime \theta^{\prime 2}}$, do not have a traditional downgradient diffusion form.''
...
doesn't contain the necessary symmetry.
We ``develop a parameterization that
possesses necessary physical properties and verify it
against large-eddy simulation (LES) and observational
data.''

\item \bibentry{MGM+00} \ \\
``The pressure transport term in the turbulence kinetic‐energy budget becomes more important as the rotation rate increases, whereas the contribution of the third‐order transport term is reduced. All terms in the buoyancy variance budget grow in amplitude as the rotation rate increases. The mean‐gradient term and the turbulent transport term are both gains that are offset by a loss to dissipation in the bulk of the CBL.''

\item \bibentry{Miro01} \ \\
``the relative importance of
the buoyancy contribution to the pressure-gradient-potential-temperature covariance decreases with increasing
rotation rate.'' \\
``the Coriolis contribution becomes more important as the rotation rate increases.'' \\
``underestimation of the pressure term in the flux budget equation and may lead to an erroneous prediction of the
vertical potential-temperature flux in convection with rotation. ''

\item \bibentry{UB03} \ \\
$k-\varepsilon$ and Mellor-Yamada are special cases.
\\
``Umlauf et al. (2003) extended the $k-\omega$
model of Wilcox (1988) ... to buoyancy affected flows''

Second equation for \\
\[
\psi - (c_\mu^0)^p k^m l^n
\]

\item \bibentry{UBH03} \ \\
Compared with $k-\varepsilon$ and M-Y. \\
``$k-\omega$ model computes correct decay rates for turbulent quantities under breaking waves'' \\
requires ``buoyancy term in the second equation is appropriately weighted''\\
Suggestion to buoyancy affected and rotating flows.

Buoyancy production term in $k$ equation is $G=-\nu_t ^\theta N^2$.

\[
\dot{\omega} = \mathscr{D}_\omega + \frac{\omega}{k}
\bigl(
    c_{\omega 1} P + c_{\omega 3} G - c_{\omega 2}\frac{f_{c\omega}}{f_{c\mu}} \varepsilon
\bigr)
\]
$c_{\varepsilon 3} = c_{\omega 3} + 1$

\item \bibentry{MY82} \ \\
``The master length scale equation we use is quite empirical'' \\
``we believe the equation we use is much to be preferred relative to the transport equation for dissipation'' \\
Level 4 model has prognostic equations for Reynolds stress, turbulent heat transport and theta variance. \\
$q^2$ is 2 TKE \\
Level 3 model, scale ``all terms in the model equations as a product of $q*3/\Gamma$ and powers of $\mathbf{a}$'' where $\Gamma$ is a length scale and $\mathbf{a}$ is a measure of anisotropy. \\
$\varepsilon=q^3/\Gamma_1$ \\

Level 2.5 neglect the material derivative and diffusion terms in prognostic equations for temperature variance. \\
Level 2 ``neglects all material derivatives and diffusion terms''

Prognostic equation for $q^2 \ell$ given.

Level 2.5 recommended: solutions of equations for $q^2$ and $q^2 \ell$. \\
Equation for $q^2 \ell$ performs better than algebraic expression for $\ell$ in level 2 model.

\item \bibentry{CIL96} \ \\
``Several illustrative applications are provided including the downward directed warm jet, the stratified mixing layer and buoyancy affected grid-turbulence decay.'' \\
Prognostic equation for $\varepsilon$ including buoyancy generation.

``Others adopt the so-called algebraic second-moment (ASM) closures
among whom Rodi (1982) has made extensive explorations of buoyancy-driven
flows. These economical approaches all work adequately when diffusion is of little
importance in the overall second-moment budget.''

\item \bibentry{BD03} \ \\
Gives $k$ and $\varepsilon$ equations with buoyancy generation term. Buoyancy generation term parameterised in terms of pressure gradients. 4 different versions of buoyancy production term given in terms of density gradients. All 4 have advantages and disadvantages.

\item \bibentry{DH70}

\item \bibentry{XDW00} \ \\
Solves TKE equation in the BL. Model cited by \cite{CHB+06}.\\
TKE equation not given but instead refers to Moeng (1984)
\item \bibentry{XDW+01} \ \\
Not about the TKE scheme

\item \bibentry{Moen84} \ \\
Gives TKE model solved for sub-grid-scale turbulence therefore mixing lengths are small

\item \bibentry{SLS+94}\ \\

\item \bibentry{HW07}

\item \bibentry{BHC97}

\end{itemize}

\section*{Boundary Layer Parameterization}

\begin{itemize}
\item \bibentry{CHB+06} \ \\
Include $k-\varepsilon$ models: \\
Duynkerke’s formulation (1988) \\ 
Mauritsen et al., 2004

\item \bibentry{Duy88} \ \\
Gives 1D $k$ and $\varepsilon$ equations with buoyancy production term, $B$. $B$ is not always present in the $\varepsilon$ equation. Closure for $B$ is gradient transport.

\item \bibentry{MSE+04} \ \\
Prognistic equaitons for TKE and potential temperature variance.

\item \bibentry{TE20} \ \\
\\If the effects of turbulence are modeled as an eddy viscosity and
diffusivity, then an idealized analysis based on the hypothesis predicts a well-known scaling for the magnitude of the eddy viscosity and diffusivity. It also predicts that the marginally stable modes should have vertical and horizontal scales comparable to the boundary layer depth.''

\item \bibentry{LBB+00}
\item \bibentry{SvH06}
\item \bibentry{HP96}
\item \bibentry{KK88}
\item \bibentry{CS03}
\item \bibentry{DWS06}
\item \bibentry{BCA76}
\item \bibentry{Loui79}

\item \bibentry{MY74}\ \\
The friction velocity is obtained from $U$ at the second level (above $z_0$). Then the heat flux at the surface is computed from MO theory. The boundary values of the turbulent moments are obtained from the TKE. Boundary values of $U$, $V$ and $\Theta$ are obtained from MO theory at first model level (above $z_0$).

%\item \bibentry{}
%\item \bibentry{}
%\item \bibentry{}
%\item \bibentry{}
%\item \bibentry{}
%\item \bibentry{}

\end{itemize}

\nobibliography{numerics}

\end{document}
