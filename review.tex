\input{utils/begin}

\section{Background}
\label{secn:background}

\subsection{A Brief Description of \elnino }

An \elnino\ event consists of a loss of balance between the easterly
trade winds in the tropical Pacific and the piling up of warm surface
waters in the western part of this ocean; the trade winds slacken or
even reverse in places and the warm surface waters spread eastwards
into the central tropical Pacific. The weaker trade winds drive less
equatorial up-welling of cold water into the cold tongue of the
eastern equatorial Pacific and less warm surface waters are piled up
in the west so the zonal SST gradient driving the
Walker circulation is weakened which feeds back into weakening the
trade winds. The thermocline also responds to the westerly wind
anomalies; it is usually sloped downwards towards the west where it is
deep and shallow in the east where water is up-welled. During an
\elnino\ the thermocline flattens so that it becomes deeper in the
eastern tropical Pacific thus reducing primary productivity and thus
fish stocks and shallower in the west. 

Deep tropical convection only occurs over the warmest
waters of the oceans (generally SST greater than about 28\de C
\cite{Web90}) so, even though the largest warm SST anomalies may occur
in the eastern tropical Pacific during \elnino , the deep tropical
convection moves into the central tropical Pacific since the warmest
waters have moved into this area. The climatologically much colder
waters of the eastern tropical Pacific only reach 28\de C occasionally
in a few places during an \elnino. Hence, droughts can occur in
Indonesia during and \elnino\ and higher than average rainfall in
Peru, although not as much as in the centre of the ocean. Diagrams of
the effects of \elnino\ around the world and a fuller description are
given in the \elnino\ theme page \cite{Nat00}. 

The evolution of an \elnino\ event is highly dependent on the seasonal
cycle. A description of the timing of a typical \elnino\ event is
given which is partly taken from Philander \cite{Phi90} and partly
from the results of the composite event calculated in the current work
(for a full description see \cite{Spe00}). The usual changes
throughout the year have some similarities with those that occur
during \elnino ; in spring the trade winds are weakest and the cold
tongue SSTs are warmest since there is less up-welling of cold water
due to the weaker trade winds. Then, normally, as the year progresses
the trade winds strengthen and the cold tongue SST reduces, to reach a
minimum in autumn. However, during an \elnino , the trade winds do not
pick up so strongly which leads to anomalously high SSTs in the
eastern tropical Pacific towards the end of the year. Philander
\cite{Phi90} notes that maximum SST anomalies due to \elnino\ occur in
the summer off the South American coast, however, the composite
derived in the present work finds maximum SST anomalies in the
following winter in the \nino\ 3 area (5\de S to 5\de N and 90\de W to
150\de W) with maximum anomalies moving from the coast to this area
between summer and winter. The anomalies weaken in the beginning of
the following year and most events have completely come to an end come
the following spring, frequently with \lanina\ conditions
following. The rates of growth and decay can be quite 
variable within different events and between different events. 

\subsection{The Dynamics of \elnino }

The description of the water piled up in the west spreading eastwards
is an over simplification of what happens during \elnino. In reality,
sub-surface Kelvin and Rossby 
waves transmit effects from one side of the ocean to the other. A
westerly wind burst may create such waves which travel along the
equator in the thermocline. A westerly wind burst occuring on the date
line may create a down-welling, equatorially trapped Kelvin wave which
travels eastwards as well as an up-welling Rossby wave which travels
westwards. 

The eastward travelling Kelvin wave may take a couple of months to
reach the South American coast where it deepens the thermocline and
raises the SSTs. Its influence on SSTs depends on how close the
thermocline is to the ocean surface. It is then deflected into two
coastally trapped 
Kelvin waves which travel northwards and southwards increasing the
SSTs along the coast. 

Since the gravest mode Rossby wave travels at one third of the speed
of the Kelvin wave, the westward travelling, upwelling Rossby wave
takes between 6 months 
to a year to reach the western boundary where it raises the
thermocline and reduces the SSTs. Thus these two waves can explain the
primary SST pattern of an \elnino\ in the tropical Pacific. They can
also explain the duration of the event; on reaching the western
boundary, the up-welling Rossby wave is reflected and becomes an eastward
travelling, upwelling, equatorially trapped Kelvin wave, travelling in
the thermocline. This may take half a year to reach the South American
coast where it raises the thermocline and reduces the SSTs, thus
bringing an end to \elnino. It also explains the tendency for \lanina\
conditions to follow an \elnino. 

Much of the current understanding of the
equatorial ocean waves that dictate the timing of \elnino\ events
\cite{MY99, McP99} came
from observations of the 1997/98 \elnino\ which was better observed
than previous events and observations included subsurface SST from
moored buoys \cite{Nat96}. 

\subsection{The Global Effects of \elnino}

The effects of \elnino\ can be broadly partitioned into the direct
effects in locations in the proximity of the tropical Pacific and the
teleconnections which are the remote effects transmitted through the
global atmospheric circulation or atmospheric bridge. 

The most
prominent teleconnections are across the (North) Pacific, North
American (PNA) region in winter. The PNA pattern consist of a Rossby
wave emanating from the anomalous convection in the tropical Pacific
which arcs northwards into the North Pacific then eastwards into North
America. This has the effect of moving the North Pacific storm track
southwards, bringing anomalous high pressure to Western Canada and
anomalously low pressure to South Eastern United States. There are
many other teleconnection patters in the Pacific region which are
outlined in \cite{TBK+98}, some of which may be studied in this
project. 

Teleconnections in the PNA sector have been modelled by many
researches, including Lau and Nath \cite{LN94, LN96} and Hoeling
et. at. \cite{HKZ97}. Atmospheric GCM experiments have been performed
imposing cycles of \elnino\ SSTs in the tropical Pacific and various
SSTs in remote ocean basins. In this way, Lau and Nath \cite{LN94,
LN96} have shown that teleconnection patterns in this region are
predominantly caused by tropical Pacific SSTs rather than
extra-tropical SSTs. Local coupling to a mixed layer ocean has been
included to show how remote SST anomalies are generated and how they
might feed back onto the atmosphere. Hoeling
et. at. \cite{HKZ97} demonstrated the differences in patterns and
locations of atmospheric responses to tropical forcing associated with
\elnino\ and \lanina, aswell as the differences in sign. 

Frequently, droughts occur in East Africa and North East Brazil during
\elnino\ and the Asian summer monsoon may be weak, although
Krishnamurthy and Goswami \cite{KG00} stipulate that this happens
during \elnino\ predominantly when the tropical Pacific is also in a
warm interdecadal phase. These tropical and subtropical effects are
associated with zonal variations in the Walker circulation combined
with local variations of regional Hadley circulation \cite{KG00}.

\input{utils/end}
