%Thesis Committee Report, December 1999.

\documentclass[11pt,a4paper]{article}

\topmargin=-21mm
\textwidth=161mm
\textheight=242mm
\oddsidemargin=0mm
\parindent=0mm
\parskip=3mm

\usepackage{natbib,psfig}
\bibstyle{danny}


\title{\textbf{\Large{Thesis Committee Report, December 1999.}}}
\author{Brenda Cohen}
\date{\textbf{Convective Scaling Theories}}

\begin{document}

\maketitle

Supervisor: Dr G. Craig\\
Thesis Committee Members: Dr G. Craig, Dr C. Thorncroft, Dr J. Thuburn

\section{Introduction}
The concept of the existence of a radiative-convective equilibrium has
proved a very instructive tool for attempting to understand convection and
its interaction with the large-scale atmosphere, particularly in the
tropics. This situation is assumed to exist when the destabilisation
of the atmosphere by radiative cooling and the consequent convective
stabilisation is constrained to be balanced by fluxes of heat and
moisture from the surface.

Whilst the radiative-convective equilibrium model is an attractively
simple view of the convective atmosphere, the approach to that
equilibrium after a perturbation in the external conditions is
somewhat less well-defined, and has rather more conceptual
subtleties. In particular, the adjustment does not occur over one
single timescale, but involves several variables adjusting at
different rates. The final equilibrium state can only be reached on
the slowest of these timescales. For example, the mass flux reaches a
very fast equilibrium with the forcing, the adjustment timescale being
of the order of two hours. (The timescale for this is probably
governed by the time it takes for a gravity wave to travel the
distance between the clouds.) On the other hand, the water vapour
adjustment is much slower, and full equilibrium is only achieved after
around twenty to thirty days. This timescale is related to the time it
takes for water vapour to be transported in the subsiding air between
the clouds, and is thus set by the radiative cooling rate. Thus, on
timescales shorter than this, the atmosphere can only be considered
to be in a \textit{partial} equilibrium (so-called
`quasi-equilibrium').

\section{Theory}
\subsection{The Scaling Relations}
In an attempt to quantify the behaviour of the equilibrium state, many
people (\cite{renno}, \cite{robe}, \cite{emanuel}, \cite{craig}, \cite{shutts}) have developed the idea of treating the convective system as a
heat engine, which transports heat from the sea surface to some lower
characteristic atmospheric temperature, and uses the work generated to
maintain the convective motions against frictional dissipation. Some
simple budgets for the equilibrium system are then used to generate
scaling relations.\\
\begin{figure}[ht!!]
\center
\mbox{\psfig{file=/home/swr98bgc/thesiscommittees/schem.eps,height=5cm,width=8cm}}
\caption{\it{Schematic of Equilibrium Convective System as a Heat
Engine}}
\end{figure}

The first budget that is invoked is an \textit{energy budget} over
the convective layer (from some reference layer, $z_b$, to a height above
the cloud tops, $z_t$). This is:
\begin{equation}
\frac{dE}{dt} + \int \Delta (flux \, of \, energy)^{z_t}_{z_b} \, dxdy = 0,
\end{equation}
\begin{tabbing}
where \= $E$ is the total volume-integrated energy, and the flux of
energy has a component, $F_{ext}$,\\
where \>$\int \Delta
F_{ext} \, dxdy \equiv$ imposed energy sink due to cooling $ =
 \int_{z_b}^{z_t} \rho C_p \dot{T} \, dz$
\end{tabbing}
In equilibrium there is a steady-state and, assuming the flux of
energy at $z_t$ is negligible, we have the relation:
\begin{equation}
F_{ext} = M_c \Delta MSE,
\end{equation}
\begin{tabbing}
where \= $M_c$ is the updraught mass flux, $MSE$ is the moist static energy = $C_pT+gz+L_vq$ \\
and \> $\Delta MSE = MSE_{up} - MSE_{down}$ at height $z_b$. 
\end{tabbing}
The second budget is an \textit{entropy budget}. Here, the sources and
sinks of entropy in the convective heat engine must
balance \textit{in equilibrium}. Two contributions to this budget come from the radiative cooling (sink of entropy) and the surface heat
fluxes (source), and their values are well-defined. This, though, leaves a
shortfall in the budget, and so there must
be other sources of entropy in the system. The most important of these
has been conjectured to be an irreversible source due to turbulent
dissipation of kinetic energy within the convective layer (\cite{emanuel}). 

A variable, D, describing the rate of conversion of kinetic energy to
heat is introduced, which can be used in the budget equation to
generate the relation:
\begin{equation}
D = \eta F_{ext},
\end{equation}
where $\eta$ is the efficiency of the heat engine. 

Defining $D \equiv M_c \,CAPE = $ work done by undiluted convective mass
transfer, leads to the expression:
\begin{equation}
CAPE \propto \Delta MSE
\end{equation}  
The last theoretical argument is the introduction of some measure of a
\textit{characteristic dissipation velocity}, $w_*$. This has
sometimes been related simply to the $CAPE$ (e.g. \cite{craig}, \cite{emanuel}), but
some (\cite{shutts}) have related it to a characteristic timescale for the dissipation
of kinetic energy by turbulence. It is unclear, though,
whether this dissipation of kinetic energy should occur simply within
the clouds, or whether gravity wave propagation ensures that the
dissipation occurs over the whole convective region. If the first is
true, then the convective velocity, $w_c$, can be found to be independent of the forcing, while
if kinetic energy dissipation occurs throughout the convective region,
then $w_c$ has a dependence on $F_{ext}$. The fractional area occupied
by convection,
$\sigma$, can also be deduced. 

i.e., \ \ \ \ \ \ \ \ \ \ \ \ \ \ \ \ \ \ \ \ \ \ \ \ \ Within Clouds \ \
\ \ \ \ \ \ \ \ \ \ \ \ \ \ \ Throughout Convective Layer 
\begin{equation}
\sigma \propto \rho^{-1} {\Delta
MSE}^{-\frac{3}{2}} F_{ext} , \ \ \ \ \ \ \ \ \ \ \ \ \ \ \ \ \ \ \  \sigma \propto \frac{{(Fext/\overline{\rho}})^\frac{2}{3}} {\Delta
MSE}
\end{equation} 
and 
\begin{equation}
w_c \propto {\Delta MSE}^\frac{1}{2}, \ \ \ \ \ \ \ \
\ \ \ \ \ \ \ \ \ \ \ \ \ \ \ \ \ \ \ \ w_c \propto \sigma^\frac{1}{2} \, {CAPE}^\frac{1}{2}
\end{equation}

\subsection{The Fluctuation-Dissipation Theorem}
Another interesting idea concerns the relationship between the
approach of the mass flux towards equilibrium, and also the
fluctuations around that equilibrium state. In particular, there
is a theory used often within the field of statistical mechanics, which says that the
adjustment timescale of a variable in a system towards equilibrium
after an applied perturbation, is the same as the autocorrelation
function of random fluctuations around that equilibrium, as long as
the applied perturbation can be considered small.

In terms of equations, this fluctuation-dissipation theorem says:
\begin{equation}
\frac{\Delta \overline{A}(t)}{\Delta \overline{A}(0)} = \frac{C(t)}{C(0)},
\end{equation}
where $A(t)$ is a dynamical variable (here the mass flux), $\Delta
\overline{A}(t) = \overline{A(t)}-\langle A \rangle$,
($\overline{A}(t)$ is the\\
 non-equilibrium ensemble average of
A, $\langle A \rangle$ is the equilibrium ensemble average), \\
and $C(t)$ is the autocorrelation function =  $\langle \delta A(0) \delta A(t) \rangle$ = $\langle A(0)A(t) -
\langle A \rangle ^2 \rangle$, \\
($\delta A(t)=A(t)-\langle A \rangle$).

The criterion that the perturbation to the equilibrium must be small,
is a statement that the theory is only valid in the \textit{linear
response region}, where the deviations from equilibrium are linearly
related to the perturbations which remove the system from its
equilibrium.
\section{Proposed Experiments}
\subsection{The Scaling Relations}
I hope to use a three-dimensional cloud-resolving model (CRM) to test
out these equilibrium theories, and the scaling relations they
suggest. 

Before outlining the series of experiments proposed, I must point out some slightly more subtle aspects which are
presently not well-understood, and which can be tested by such
a study. One of the biggest unknowns is the dependence of $\Delta
MSE$ on the forcing. It is well-known that the convective mass flux
adjusts to a perturbation to the forcing on a very short timescale
(around two hours), while the domain-averaged $MSE$ (governed most
strongly by the slow water vapour adjustment) shows \textit{only} a
very slow adjustment towards equilibrium (\mbox{\cite{adrian}}).\\
\begin{figure}[ht!!]
\center
\mbox{\psfig{file=/home/swr98bgc/thesiscommittees/MSEtot.eps,height=5cm,width=8cm}}
\caption{\it{Timeseries of Domain-Averaged Moist Static Energy}}
\end{figure}
 But the adjustment of
$\Delta MSE$ is a little less clear-cut. This adjustment is likely to
be quite closely tied to the response of the boundary layer $MSE$,
since the updraughts transport boundary layer air upwards within clouds.

This possible dependence of $\Delta MSE$ on the forcing has important
consequences for the nature of the partial equilibrium state. If it
does \textit{not} show a rapid adjustment then, from equations 2 and 4,
it would be expected that $M_c$ would be directly proportional to the
forcing, and $CAPE$ independent of it. If the opposite were true,
$M_c$ and $\Delta MSE$ would be correlated on the short convective
timescale, and their equilibrium values would depend in some more
complicated way on $F_{ext}$.

Another point concerning $\Delta MSE$ is the precise definition of
the downdraught $MSE$. Most people have assumed that the contribution of
penetrative downdraughts to $\Delta MSE$ can be considered negligible,
and the downdraught $MSE$ is therefore taken as the $MSE$ of the
slowly subsiding air around each cloud. Of course, it is convenient to
be able to ignore these downdraughts for the purposes of convective
parameterization schemes; it would be difficult to find a closure on
the downdraught $MSE$, while the downdraught mass flux is usually
crudely assumed to be $0.2$ times that of the updraught. However, it
seems likely that such a neglect must be unjustified, certainly in
regions of convective organisation.

Another issue to be investigated is that of the vertical level at
which the scaling relations hold. Shutts and Gray derived their
relations at the top of the shallow convective layer, which is assumed
to be the MSE minimum (see Figure 3), but it is unclear whether this is a correct definition, or
whether the scaling relations hold at different heights (e.g. others
assume they should be taken at the top of the sub-cloud layer).

With these questions in mind, a set of experiments is envisaged:
\begin{itemize}
\item A series of perturbation experiments will be run, in which $F_{ext}$ and
$\Delta MSE$ are varied independently. Three or four variations of
each parameter will be chosen. The dependences of $M_c$, $CAPE$, $w_c$
and $\sigma$ will be investigated, in comparison with the relations
outlined in section 2.1.
\item The timeseries of $M_c$ and $\Delta MSE$ after perturbations have
been applied can be plotted, to check whether a rapid adjustment is
seen in both.
\item The vertical profiles of $MSE_u$, $MSE_d$ and $MSE_s$ and also
their associated mass fluxes can be plotted for all the different
forcings. The scaling of downdraught mass flux with the updraught
value can be investigated, and some feel for a possible closure on
$MSE_d$ can perhaps be found.
\item The values of the quantities in the scaling relations can be
calculated at different vertical levels, to ascertain the height(s) at
which these relations hold.\\
\end{itemize}
\begin{figure}[h!!]
\center
\mbox{\psfig{file=/home/swr98bgc/thesiscommittees/MSEhgt.eps,height=5.2cm,width=7.8cm}}
\caption{\it{Vertical profile of Total Moist Static Energy}}
\end{figure}
\subsection{The Fluctuation-Dissipation Theorem}
Some obvious questions arise from the fluctuation-dissipation theorem,
which should also be verifiable within the set of experiments already proposed:
\begin{itemize}
\item Can this fluctuation-dissipation theorem hold true for the partial
equilibrium system being considered here? If so, for what range of
values of perturbation can the linear regime be said to exist? The
perturbation experiments to $F_{ext}$ can be used to compute
separately the changes
in relaxation time back to equilibrium and the autocorrelation function.
\item Is this theory valid at all vertical heights separately? Or will
it only be satisfied when domain-averages are considered? 
\item It would be interesting to see whether any physical explanation
can be attached to the adjustment timescale. It is wondered whether
this adjustment is related to the cloud-spacing or to the cloud
lifetime. If it were related to the cloud-spacing, it would be
interesting to see how the adjustment timescale varied with the
fractional area occupied by convection.
\end{itemize}

\section{Experimental Issues}
I have managed to set up the CRM and have added most of the
diagnostics which will be required. 

However, there are some very important practical issues which must be
addressed. Firstly, all the relations being tested are valid only for
idealised ensemble averages. Because of computational
limitations, however, this ensemble average will never be obtained; in practice, it will be essential to take a combined spatial and
temporal average, to ensure a sufficient number of clouds are included. The range over which the time average may be taken
must be considered carefully: if it is, say, less than two
hours, the averaging will be over strongly correlated fields and will
yield no truly independent measurements. If, however, the averaging
period is greater than around twelve hours, the
background $MSE$ may have drifted significantly enough to have altered
the true equilibrium average mass flux value.

Sensitivity tests must be done to assess the convergence to the
equilibrium average, by plotting the distribution of cloud mass
fluxes, which should form a Gaussian around the ensemble average. In
this limit, $\langle M_c \Delta MSE \rangle$ should be
well-approximated by $\langle M_c \rangle \langle \Delta MSE \rangle$,
which is an assumption used in the derivation of the scaling
relations. If the domain used is too small, problems will occur, and
so a larger domain (or at least a higher cooling rate, to pack the
clouds closer) must be used.

A final point is that to test the fluctuation-dissipation theorem
requires an instantaneous spatial average to be taken in the approach
to equilibrium (and also for the autocorrelation). This sets a lower
limit to the horizontal domain size possible.

The perturbation experiments must also be designed carefully. In
particular, it is suggested that the model be spun up into an
equilibrium state for around fifteen to twenty days, to ensure that
the water vapour field has adjusted from its initial homogeneous unrealistic
distribution to a more realistic one. After this spin up, the
perturbations are then applied and the model is allowed to adjust on
the short two hour timescale before the scaling relations can be
tested. Meanwhile, the relaxation towards equilibrium must be used to
test the fluctuation-dissipation theorem.

Finally, the varying of $F_{ext}$ and $\Delta MSE$ independently must
be considered. Varying $F_{ext}$ simply requires changing the
radiative cooling rate, but adjusting $\Delta MSE$ is harder. It is
suggested that to adjust $RH^* - RH$ by a fixed amount at every
grid-point except within clouds (where $RH \equiv RH^*$) would be a
suitable way to effect the perturbation.

Thus, it would seem that the main experiments should be carried out on
a large domain at quite high resolution. The exact numbers must still
be checked, but it would seem prudent to run the model on a grid at
least $256 \times 256\, km$ with a resolution of at least $1\, km$
(perhaps even $500\, m$; the sensitivity to resolution can be
checked). Because of the long spin up time, it will perhaps be
necessary to run the approach to equilibrium on a much smaller, coarse
resolution grid, and then copy the domain several times and
interpolate down before adding the perturbations.

\bibliographystyle{natbib}
\bibliography{tcd99}   
\end{document}